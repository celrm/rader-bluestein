% https://github.com/martinhelso/UiB


\documentclass[spanish]{beamer}
\usepackage[spanish]{babel}
\usetheme{UiB}


\usepackage[utf8]{inputenx} % For æ, ø, å
\usepackage{csquotes}       % Quotation marks
\usepackage{microtype}      % Improved typography
\usepackage{amssymb}        % Mathematical symbols
\usepackage{mathtools}      % Mathematical symbols
\usepackage[absolute, overlay]{textpos} % Arbitrary placement
\setlength{\TPHorizModule}{\paperwidth} % Textpos units
\setlength{\TPVertModule}{\paperheight} % Textpos units
\usepackage{tikz}
\usetikzlibrary{overlay-beamer-styles}  % Overlay effects for TikZ


\author{María Arranz y Celia Rubio}
\title{Transformada rápida de Fourier}
\subtitle{Métodos de Rader y Bluestein}
\date{Rader y Bluestein}

\usepackage{faktor}
\newcommand{\DFT}{\text{DFT}}
\newcommand{\IDFT}{\text{IDFT}}
\usepackage{cancel}

\begin{document}


\section{Introducción}
\begin{frame}{Índice}
    \tableofcontents
\end{frame}

\begin{frame}{Introducción}
    Sea el polinomio $P(x)=\sum\limits_{i=0}^{N-1}a_i\cdot x^i$.

    \begin{definition}
    $\DFT(P,\xi) = \left(P(1),P(\xi),P(\xi^2)...\right) = D(\xi)\cdot (a_0,a_1,a_2...)$\\[\bigskipamount]
    donde $D(\xi)_{(i,j)}=\xi^{ij}$ para $i,j\in\{0,...,N-1\}$. 
    \end{definition}
    
    Es decir,  para $j\in\{0,...,N-1\}$:
    \begin{equation*}
    \DFT(P,\xi)_{(j)}=\sum\limits_{k=0}^{N-1} a_k \cdot\xi^{jk}
    \end{equation*}
    
    En clase hemos visto Cooley–Tukey para $N$ potencia de $2$:
    \begin{equation*}
        \DFT(P,\xi)_{(j)}=P(\xi^j)=P_{par}(\xi^{2j})+\xi^j\cdot P_{impar}(\xi^{2j})
    \end{equation*}
\end{frame}

\section{Método de Rader}

\begin{frame}{Método de Rader (I)}\medskip

    Si $N$ primo $\implies$
    $\left(\faktor{\mathbb{Z}}{N\mathbb{Z}}\right)^*$ es el grupo cíclico $C_{N-1}$ de orden $N-1$.\\[\smallskipamount]
    \begin{itemize}
        \item Tiene un generador $g$ de orden $N-1$.
        \item $g^{-1}$ también es un generador.
    \end{itemize}\bigskip
    
    Retomando la fórmula:
    \begin{equation*}
        \DFT(P,\xi)_{(j)}=\sum\limits_{k=0}^{N-1} a_k \cdot\xi^{jk}
    \end{equation*}
    
    En concreto, $\DFT(P,\xi)_{(0)}=P(1)=\sum\limits_{k=0}^{N-1} a_k $.\\[\bigskipamount]
    
    El resto de índices $j\in\{1,...,N-1\}$ los vemos dentro de $C_{N-1}$. 
\end{frame}

\begin{frame}{Método de Rader (II)}
    \begin{equation*}
        \DFT(P,\xi)_{(j)}=a_0+\sum\limits_{k=1}^{N-1} a_k \cdot\xi^{jk},\quad j\in\{1,...,N-1\}
    \end{equation*}
    Los índices $k$ también los vemos dentro de $C_{N-1}$.\\[\smallskipamount]
    \begin{itemize}
        \item Escribimos $k$ como potencia de $g$:\qquad
            $k=g^{q_k}$
        \item Escribimos $j$ como potencia de $g^{-1}$:\qquad 
            $j=(g^{-1})^{p_j}=g^{-p_j}$
    \end{itemize}\medskip
    
    $j\mapsto p_j$ y $k\mapsto q_k$ son biyecciones, con $p_j,q_k = 0,...,N-2$.\bigskip
    
    Reescribimos la fórmula (quito subíndices para aligerar):
    \begin{equation*}
        \DFT(P,\xi)_{(g^{-p})}=a_0+\sum_{q=0}^{N-2} a_{g^{q}} \cdot\xi^{g^{-(p-q)}}
    \end{equation*}
\end{frame}

\begin{frame}{Método de Rader (III)}
    \begin{equation*}
        \DFT(P,\xi)_{(g^{-p})}=a_0+\sum_{q=0}^{N-2} a_{g^{q}} \cdot\xi^{g^{-(p-q)}}
    \end{equation*}
    Es la convolución cíclica de $u(n) = \xi^{g^{-n}}$ y $v(m)=a_{g^m}$.
    \begin{equation*}
        (u*v)_p = \sum\limits_{\substack{n+m\equiv p\ (\text{mod}\ N-1)\\ n,m=0}}^{N-2}u(n)\cdot v(m)
        =(\IDFT\left(\DFT(u)\cdot\DFT(v)\right))_p
    \end{equation*}
    
    ¿Cómo lo calculamos? Necesitamos:
    \begin{itemize}
        \item \textbf{Alargar con ceros} $u$ y $v$ para que tengan longitud potencia de 2.
        \item Que la solución resultante vuelva a tener longitud $N-1$.
    \end{itemize} 
\end{frame}

\section{Rellenado con ceros}

\begin{frame}{Rellenado con ceros (I)}\medskip
    Sean $A$ y $B$ de longitud $M$. 
    
    Rellenamos con ceros hasta $\ell = 2^k > 2M$: 
    \begin{equation*}
        \mathcal{A} = A + [0]*(\ell - M)\ \qquad 
        \mathcal{B} = B + [0]*(\ell - M)
    \end{equation*}
    Hacemos la convolución de las extensiones $\mathcal{A}$ y $\mathcal{B}$:
    \begin{equation*}
        (\mathcal{A}*\mathcal{B})_k = \sum_{i+j=k} \mathcal{A}_i \cdot \mathcal{B}_j + \cancelto{0}{\sum_{i+j=k+\ell} \mathcal{A}_i \cdot \mathcal{B}_j}
    \end{equation*}
    El segundo sumatorio tiene todos sus términos nulos. 
    
    Si $i<M$ entonces $j>=M$, y viceversa:
    \begin{equation*}
        j = k+\ell-i > k+2M-M= k + M >= M
    \end{equation*}
    
\end{frame}

\begin{frame}{Rellenado con ceros (II)}\medskip
    Además, si $i$ o $j >= M$ entonces el sumando es 0, así que los índices solo llegan hasta $M-1$.
    \begin{equation*}
        (\mathcal{A}*\mathcal{B})_k = \sum_{i+j=k} \mathcal{A}_i \cdot \mathcal{B}_j = \sum_{i+j=k} A_i \cdot B_j
    \end{equation*}
    ¿Cómo se recupera entonces la convolución original?
    \begin{equation*}
        (A*B)_k = \sum_{i+j=k} A_i\cdot B_j + \sum_{i+j=k+M} A_i\cdot B_j
    \end{equation*}
    \begin{equation*}
        = (\mathcal{A}*\mathcal{B})_k + (\mathcal{A}*\mathcal{B})_{k+M}
    \end{equation*}
    
\end{frame}


\section{Método de Bluestein}

%\begin{frame}{Método de Bluestein}
%\url{https://en.wikipedia.org/wiki/Chirp_Z-transform\#Bluestein.27s_algorithm}
%\url{https://ccrma.stanford.edu/~jos/st/Bluestein_s_FFT_Algorithm.html}
%\end{frame}

\begin{frame}{Método de Bluestein (I): CZT}
La transformada Z chirp (CZT) 
\begin{itemize}
    \item Generalización de la DFT, tomando puntos sobre arcos espirales en el $Z$-plano (lineas rectas en el $S$-plano)
\end{itemize}
    \begin{definition}
    Sea $X_k = CZT_k$ para $A$, $W$ y $M$ dados:\\
    $X_k = \sum\limits_{n=0}^{N-1} x(n) z^{-n}_{k}\ \ \ \ \ z_k=A \cdot W^{-k}, k = 0,\dots, M-1$\\
    $A$ número complejo de partida, W ratio complejo entre puntos, M número de puntos a calcular.
    \end{definition}
\end{frame}

\begin{frame}{Bluestein (II)}
\begin{equation*}
\DFT(P,\xi)_{(j)}=\sum\limits_{k=0}^{N-1} a_k \cdot\xi^{-jk}
\end{equation*}
Reemplazando $jk$ por la siguiente identidad:
\begin{equation*}
    jk = \frac{k^2+j^2-(j-k)^2}{2}
\end{equation*}
Se obtiene:
\begin{equation*}
    \DFT(P,\xi)_{(j)}=\sum\limits_{k=0}^{N-1} a_k \cdot\xi^{( \frac{k^2+j^2-(j-k)^2}{2})} = \sum\limits_{k=0}^{N-1} a_k \cdot\xi^{\frac{-k^2}{2}} \cdot\xi ^ {\frac{-j^2}{2}} \cdot\xi ^{\frac{(j-k)^2}{2}} =
\end{equation*}
\begin{equation*}
     = \boxed{\xi ^{\frac{-j^2}{2}} \cdot \sum\limits_{k=0}^{N-1} (a_k \cdot\xi ^ {\frac{-k^2}{2}}) \cdot \xi^{ \frac{(j-k)^2}{2}} }
\end{equation*}
\end{frame}


\begin{frame}{Bluestein (III)}
\begin{equation*}
      \DFT(P,\xi)_{(j)}= \xi ^{\frac{-j^2}{2}} \cdot \sum\limits_{k=0}^{N-1} (a_k \cdot\xi ^ {\frac{-k^2}{2}}) \cdot \xi^{ \frac{(j-k)^2}{2}}
\end{equation*}

Que es una convolución de las secuencias $u(n)$ y $v(n)$ dadas por:
\begin{equation*}
    u(n)= a_n \cdot\xi ^ {\frac{-n^2}{2}}
\end{equation*}
\begin{equation*}
    v(n) = \xi ^ { \frac{n^2}{2}}
\end{equation*}
Con el resultado de la convolución multiplicado por $N$ factores $v^{*}(j)$, $j\in\{0,...,N-1\}$. Esto es:

\begin{equation*}
      \DFT(P,\xi)_{(j)} =  v^{*}(j) \left(\sum\limits_{k=0}^{N-1} u(k) \cdot v(j-k)\right) =  v^{*}(j)(u * v)_j
\end{equation*}

\end{frame}
\begin{frame}{Bluestein (IV)}
    Por el teorema de convolución, se tiene, por tanto:
    
\begin{equation*}
    \DFT(P,\xi)_{(j)} =  v^{*}(j)(u * v)_j =   v^{*}(j) \cdot (\IDFT (\DFT(u) \cdot \DFT(v)))_j
\end{equation*}
\begin{equation*}
    j\in\{0,...,N-1\}
\end{equation*}

De manera más abreviada:
\begin{equation*}
    \DFT(P,\xi) =  v^{*} \cdot (u * v) =   v^{*} \cdot \IDFT (\DFT(u) \cdot \DFT(v))
\end{equation*}

Con lo que se obtiene una implementacion de complejidad $\mathcal{O}(n\log{}n)$
\end{frame}
\end{document}